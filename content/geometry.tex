\section{Geometry}

\subsection{Пересечение прямых}

\begin{equation*}
	AB \coloneqq A - B; CD \coloneqq C - D
\end{equation*}
\begin{equation*}
	(A \times B \cdot CD.x - C \times D \cdot AB.x : A \times B \cdot CD.y - C \times D \cdot AB.y : AB \times CD)
\end{equation*}

\subsection{Касательные}

Точки пересечения общих касательных окружностей с центрами в $(0, 0)$ и $(x, 0)$ равны $\frac{x r_1}{r1 \pm r2}$.
$x$ координата точек касания из $(x, 0)$ равна $\frac{r^2}{x}$.

\subsection{Пересечение полуплоскостей}

Точно так же, как в выпуклой оболочке, но надо добавить bounding box (квадратичного размера относительно координат на входе) и завернуть два раза.
Ответ можно найти как подотрезок от первой полуплоскости типа true до нее же самой на втором круге.
Проверку на вырожденность лучше делать простой проверкой пары-тройки точек из предполагаемого ответа. Стоит быть аккуратнее с точностью.

\subsection{Формулы}

Площадь поверхности сферы $4 \pi R^2$. Обьем шара $\frac43 \pi R^3$. Площадь шапки $2 \pi R h$, обьем $\frac{\pi h (3a^2+h^2)}{6}$, где $h$~--- высота, $a$~--- радиус шапки. Объем тетраэдра $\frac16$ на определитель. В общем случае площадь $S_{n - 1}$ и объем $V_n$ шарика в $\mathbb{R}^n$ можно найти по формуле $S_{n - 1} = n C_n R^{n - 1}$, $V_n = C_n R^n$, где $C_n = \frac{\pi^{\frac{n}{2}}}{\Gamma(\frac{n}{2} + 1)}$. Или альтернативно $C_{2k} = \frac{\pi^k}{k!}$, $C_{2k + 1} = \frac{2^{k + 1} \pi^k}{(2k + 1)!!}$. Также, должны быть верны формулы $\frac{V_n}{S_{n - 1}} = \frac{R}{n}$, $\frac{S_{n + 1}}{V_n} = 2 \pi R$.
