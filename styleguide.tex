\documentclass[12pt,a4paper,russian]{article}

\sloppy

\usepackage[small]{complexity}
\usepackage{csquotes}
\usepackage[russian,english]{babel}
\usepackage{amssymb}
\usepackage{amsmath}
\usepackage{amsthm}
\usepackage{verbatim}
\usepackage{comment}
\usepackage{bookmark}
\usepackage{mathtools}
\usepackage{makecell,xcolor}
\usepackage{tikz}
\usepackage[backend=biber,style=alphabetic]{biblatex}

\usepackage{forloop}

\usepackage[left=2cm,right=2cm,top=2cm,bottom=2cm,bindingoffset=0cm]{geometry}
\usepackage{fancybox,fancyhdr}
\usepackage{lastpage}
\usepackage{titling}

\usepackage{fontspec}

\setmainfont[Ligatures={TeX,Historic}]{CMU Serif}
\setsansfont{CMU Sans Serif}
\setmonofont{Fira Code}[
	Contextuals=Alternate  % Activate the calt feature
]
\usepackage{listings}
\usepackage{lstfiracode}
\lstset{
	style=FiraCodeStyle,
    basicstyle=\ttfamily,   % Use \ttfamily for source code listings
    numbers=left,
	breaklines=true,
    postbreak=\mbox{\textcolor{red}{$\hookrightarrow$}\space}
}

\lstset{tabsize=4}

\DeclarePairedDelimiter\set{\lbrace}{\rbrace}
\DeclarePairedDelimiter\abs{\lvert}{\rvert}
\DeclarePairedDelimiter\ceil{\lceil}{\rceil}
\DeclarePairedDelimiter\floor{\lfloor}{\rfloor}

\newcommand{\map}[2]{\set{#1 \mid #2}}

\def\sqgrid{
	\newpage

	\begin{figure} \centering \tikz{
		\foreach \x in {0,...,50} { % вот тут количество строк
			\draw[color=gray,opacity=0.45] (0, 0.5 * \x cm) -- (17, 0.5 * \x cm);
		}
		\foreach \x in {0,...,34} { % вот тут количество столбцов
			\draw[color=gray,opacity=0.45] (0.5 * \x cm, 0) -- (0.5 * \x cm, 25);
		}
	} \end{figure}
}

\def\dotgrid{
	\newpage

	\begin{figure} \centering \tikz{
		\foreach \x in {0,...,50} {
			\foreach \y in {0,...,34} {
				\fill[fill=gray,opacity=0.45] (0.5 * \y cm, 0.5 * \x cm)
					node{\textcolor{gray}{.}};
			};
		}
	} \end{figure}
}

\begin{document}

\selectlanguage{russian}

\tableofcontents

\section{CLion}

\subsection{Settings}

\begin{itemize}
	\item[$\rightarrow$] Editor
	      \begin{itemize}
		      \item[$\rightarrow$] Code Style $\rightarrow$ C/C++
		            \begin{itemize}
			            \item[$\rightarrow$] Wrapping and Braces
			                  \begin{itemize}
				                  \item[$\rightarrow$] Braces placement $\rightarrow$ все next line
				                  \item[$\rightarrow$] `if()' statement $\rightarrow$ `else' on new line
			                  \end{itemize}
			            \item[$\rightarrow$] Tabs and Indents $\rightarrow$ Use tab character $\rightarrow$ Smart tabs
		            \end{itemize}

		      \item[$\rightarrow$] Font $\rightarrow$ Font: Fira Code $\rightarrow$ Enable font ligatures (делает красивый шрифт)
	      \end{itemize}

	\item[$\rightarrow$] Build, Execution,\ldots $\rightarrow$ CMake $\rightarrow$ тыкнуть на плюсик, список должен стать из Debug и Release, через некоторое время они подтянутся в выпадайку при компиляции
\end{itemize}

\subsection{Shortcuts}

\begin{itemize}
	\item Ctrl+Alt+L --- форматирование, можно тыкать постоянно
	\item Shift+F10 --- запуск
	\item Shift+F9 --- запуск с дебаггером (надо не забыть сменить на дебаг в выпадайке)
	\item Ctrl+F9 --- компиляция
	\item Ctrl+F8 --- breakpoint
	\item Ctrl+F7/F8/F9 при дебаге step into/step over/resume
	\item Ctrl+F2 останавливает программу
	\item Shift+F6 переименовывает переменную и все ее вхождения
	\item Ctrl+B переходит к объявлению функции/переменной/класса под курсором или переходит в файл
	\item Alt+вертикальные стрелочки ходит по функциям
	\item Alt+горизонтальные стрелочки ходит по файлам
	\item Alt+цифра открывает/закрывает соответствующую менюшку
	\item Если дважды нажать на Ctrl, зажать и потыкать на стрелочки вверх/вниз, это сделает мультикурсор, а Escape его потом обратно уберет
\end{itemize}

\section{Шаблон}

\lstinputlisting[language=C++]{code/template.cpp}

\section{Общее}

Избегаем глобальных переменных, кроме констант, кооторые обязательно должны быть const (исключение --- mt). Если есть какая-то глобальная динамика или еще что-нибудь такое, заводим для ее внутренностей класс или хотя бы функцию со static переменными.

DRY: Все константы в коде, которые используются несколько раз, должны быть обернуты в константы, код, который используется несколько раз, должен быть обернут в функцию или хотя бы в лямбду

KISS: Избегаем неиспользуемого и длинного кода. Если написали какую-то ерунду и поняли, что ее можно было написать вдвое короче и проще, лучше переписать, потому что это все равно потом дебажить.

Базу индукции в рекурсии лучше разбирать в начале функции, так эту проверку придется писать только один раз и будет легче проверять.

Переменные лучше называть так, чтобы об их назначении можно было догадаться только по их объявлению, писать длинные названия переменных в коде поможет автодополнение (Ctrl+пробел).

Если есть, например, таблица размера $n \times m$ и она хранится в каком-то двумерном векторе, то для получения ее размеров лучше использовать метод size этого самого вектора, а не константы $n$ и $m$. Аналогично, если какой-то фор должен идти до конца массива, лучше условие написать до конца массива, а не до $n$. Эту убережет от проблем, когда выясняется, что жизнь, например, будет лучше, если в лабиринт добавить границу, или в граф --- пару фиктивных вершин. В идеале, константы типа размера входа должны использоваться только при считывании.

\end{document}
